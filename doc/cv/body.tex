I'm a theoretical ecologist with a solid background in Computer Science and Machine Learning. During the past ten years, I studied the stability, critical transitions, and structural features of the networked ecological systems by applying the Random Matrix Theory and Ordinary/Stochastic Differential Equation models. At the same time, I also gained strong skills in machine learning and software development. I'm very professional in Python, Python-based deep learning frameworks such as Pytorch and Jax, and Python-based data analysis tools such as NumPy, Pandas, Matplotlib, etc. I'm always passionate in applying Machine Learning (Deep Leaning) in ecology.

Transformer models are a good starting point to apply deep learning in ecological modeling. The self-attention mechanism in the transformer models is essentially similar to the interactions among species. Each species has traits or phenotypes, represented by feature vectors, that determine its interacting types and strength with other species. The feature vectors of species are similar to the key and query vectors in the self-attention mechanism. They are so analogous to each other that I am trying to introduce the replicator dynamics in the evolutionary theory into transformer to improve it.

After we have a species interaction model, the species abundance and geographic distributions estimate a target probability density based on empirical samples. The deep generative models, such as the normalizing flows and the diffusion model, have achieved state-of-the-art results in probability density estimation. 

After we have a species interaction model and a density estimation model, a non-linear and networked dynamic system is also necessary to predict ecological dynamics such as critical transitions. In this aspect, advanced deep learning methods, such as autoencoders and reinforcement learning, can be used.

I plan to develop a deep learning platform for quantitative biodiversity dyanmics with which we can understand how species interact with each other and respond to the environment, and can also predict possible critical transitions in ecological dynamics.

% Finally, I personally love the Netherlands. From childhood, I knew that the Netherland was the land of tulips and windmills. 
I believe I'm the best canditate for this position. I ensure that I will bring new sight to biodiversity protection in the AI/DL era. I'm looking forward to hearing from you.
